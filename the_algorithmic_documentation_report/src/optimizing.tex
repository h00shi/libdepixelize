The vector arts generated by the previous steps are fine if you apply small
factors of zoom, but after magnifying the image several times some staircasing
artifacts become clear. To remove the staircasing artifacts, you need to
optimize the B-Spline's control points (do \textbf{not} confuse the control
points of the B-Spline with the control points of the Bézier curves, or you will
kill continuity and the result will look awful).

Kopf-Lischinski proposes the below \emph{energy} formula as the term to
optimize:

$$E^{(i)} = E^{(i)}_s + E^{(i)}_p$$

$E^{(i)}_s$ stands for smoothness energy and $E^{(i)}_p$ stands for positional
energy.

The smoothness energy formula given on the paper is:

$$E^{(i)}_s = \int\limits_{s \in r(i)} |k(s)| \mathrm{d}s$$

$k(s)$ stands for curvature at point $s$. The curvature formula for parametric
equation is given below:

$$k = \frac{x'y'' - y'x''}{(x'^2+y'^2)^{\frac{3}{2}}}$$

If you've read \href{http://pomax.github.io/bezierinfo/}{Pomax's \emph{A Primer
on Bezier Curves}}, then you already demystified the $k(s)$ formula and you are
ready to integrate it, but the Pomax's text is abstract enough for Bézier curves
of any order, then I'll help you further giving the required formulas for
quadratic Bézier curves:

$$Bezier(2, t) = (1-t)^2 w_0 + 2 (1-t) t w_1 + t^2 w_2$$

$$Bezier'(2, t) = 2 (1-t) (w_1-w_0) + 2 t (w_2-w_1)$$

$$Bezier''(2, t) = 2 (w_2 - 2 w_1 + w_0)$$

Replace the $w_i$ term by the node position on the dimension you want to
compute.

Numerical integration is easy and you can use any method you like. There is a
good text about numerical integration on
\href{http://jeremykun.com/2012/01/08/numerical-integration/}{Math $\cap$
Programming} and after you find the methods names, Wikipedia is good enough. If
you want a deep understanding on this topic, \emph{Heath's Scientific Computing}
book is a good start. You should use the $\{0..1\}$ range for the integration.

The positional energy formula is given on the paper and uses the
\href{http://en.wikipedia.org/wiki/Norm_(mathematics)}{norm} concept.
